\documentclass{article}
\usepackage[a4paper, total={6in, 10in}]{geometry}

\usepackage{mymath,natbib}


\title{Lower Bounds for RL}
\author{Jingfeng Wu}
\date{Created: July, 2020\\ Last updated: \today}

\begin{document}
\maketitle

\section{Preliminary}
\subsection{Yao's principle}

\begin{thm}[Yao's principle]
Let $X$ be a finite set of inputs, $\Acal$ be a finite set of deterministic algorithms. Let $R$ be a random algorithm, i.e., an algorithm that is randomly drawn from set $\Acal$ according to certain distribution $\Rcal$. Let $\Dcal$ be a distribution over the input set $X$.
Let $\cost(\Rcal, x) = \expect[A\sim \Rcal]{\cost(A,x)}$.
Let $\cost(A, \Dcal) = \expect[x\sim \Dcal]{\cost(A,x)}$.
Then we have
\begin{align*}
    \min_{\Rcal} \max_{x\in X} \cost(\Rcal, x) = \max_{\Dcal} \min_{A\in\Acal} \cost (A, \Dcal).
\end{align*}
\end{thm}
\begin{cor}
    \begin{align*}
        \max_{x\in X} \cost(\Rcal, x) \ge \min_{A\in\Acal} \cost (A, \Dcal).
    \end{align*}
\end{cor}

\subsection{Fano's inequality}

\section{Regret lower bound}
The lower bound is due to~\citep{auer2009near}.


\section{PAC lower bound}
The lower bound is due to~\citep{azar2013minimax}.

\section{PAC lower bound for reward-free exploration}
The lower bound is due to~\citep{jin2020reward}.


\bibliographystyle{abbrvnat}
\bibliography{ref}
\end{document}
